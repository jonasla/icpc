\section{Matematica}

\subsection{PotLog}
\begin{code}
const tint nmod = 1000000007; // o el primo que deseamos
tint potLogMod (tint x, tint y) // Calcula: (x^y) mod nmod
{
	tint ans = 1;
	while (y > 0)
	{
		if (y % 2)
			ans = (x * ans) % nmod;
		x = (x * x) % nmod;
		y /= 2;
	}
	return ans;
}
tint invMod(tint a) // nmod PRIMO. Devuelve b tal que: (a*b) = 1 (mod nmod) 
{
	return potLogMod(a,nmod-2);
}
\end{code}

\subsection{Criba}
\begin{code}
const tint maxN = 1000500;
tint p[maxN + 1] = {1, 1};
tint phi[maxN]; 

map<tint,tint> factorizar (tint n)
{
	map<tint,tint> f;
	while (n > 1) 
	{ 
		f[p[n]]++;
		n /= p[n]; 
	}
	return f;
}
// USO:
int main()
{
	// CRIBA COMUN : (p[n] = mayor primo que divide a n (n >= 2) )
	for (tint i = 1; i <= maxN; ++i)
		if (p[i] == 1)
			for (tint j = i; j <= maxN; j += i)
				//if (p[j] == 1 or i == 1) // Con esta linea da el menor primo
				p[j] = i;
	// CALCULA PHI(N): #Coprimos con N
	for (tint i = 0; i < maxN; i++) 
		phi[i] = i;	
	for (tint i = 1; i < maxN; i++)
	for (tint j = 2 * i; j < maxN; j += i)
		phi[j] -= phi[i];
	return 0;
}
\end{code}

\subsection{Euclides Extendido}
\begin{code}
tint gcd (tint a, tint b, tint &x, tint &y)
{
	if ( a == 0 ) 
	{
		x = 0 ; y = 1 ;
		return b ;
	}
	tint x1, y1 ;
	tint d = gcd (b % a, a, x1, y1) ;
	x = y1 - ( b / a ) * x1 ;
	y = x1 ;
	return d;
} 

// Nota si gcd(a,m) == 1 => 1 = a*x+m*y => 1 = a*x (mod m)
// O sea, que "x" es el inverso de "a" modulo "m"

\end{code}

\subsection{Teorema Chino del Resto}

\begin{code}

pair<tint,tint> tcr (vector<tint> &r, vector<tint> &m) // x = r_i (mod m_i)
{
	tint p = 0, q = 1, n = r.size();
	forn(i,n)
	{
		p = modulo(p-r[i],q);
		tint x,y;
		tint d = gcd(m[i],q,x,y);
		if (p % d)
		return {-1,-1}; // sistema incompatible
		q = (q / d)*m[i];
		p = modulo(r[i]+m[i]*x*(p/d), q);
	}
	return {p,q}; // x = p (mod q)
}


\end{code}


\subsection{Eliminacion Gaussiana (Código ruso)}
\begin{code}
// Declarar EPS e INF al principio adecuadamente.
tint gauss ( vector < vector < double > > a, vector < double > & ans ) {
	tint n = a. size ( ) ;
	tint m = a[0].size() -1;
	vector<tint> where (m, -1);
	for ( tint col = 0, row = 0 ; col < m && row < n; ++col)
	{
		int sel = row ;
		for ( tint i = row ; i < n ; ++i)
			if ( abs (a[i][col]) > abs (a[sel][col]) )
				sel = i ;
		if ( abs(a[sel][col]) < EPS )
			continue ;
		for ( tint i = col ; i <= m ; ++i)
			swap (a[sel][i], a[row][i]);
		where[col] = row;
		for ( tint i = 0 ; i < n ; ++i)
			if ( i != row ) 
			{
				ldouble c = a[i][col] / a[row][col] ;
				for ( tint j = col ; j <= m ; ++ j)
					a[i][j] -= a[row][j] * c;
			}
		++row;
	}
 
	ans.assign(m, 0);
	for ( tint i = 0 ; i < m ; ++i )
		if ( where[i] != - 1 )
			ans [i] = a[where[i]][m] / a [where[i]][i];
	for ( tint i = 0 ; i < n ; ++i ) 
	{
		ldouble sum = 0 ;
		for ( tint j = 0 ; j < m ; ++j )
			sum += ans[j]*a[i][j];
		if ( abs(sum - a[i][m] ) > EPS )
			return 0 ;
	}
 
	for ( int i = 0 ; i < m ; ++ i )
		if ( where[i] == - 1)
			return INF ;
	return 1 ;
}

// 0 -> No hay solucion
// 1 -> Hay solucion unica. La devuelve en "ans"
// INF -> Hay infinitas soluciones

\end{code}

\subsection{Rabin-Miller}
\begin{code}
// USA: "PotLog", pero pasandole el modulo como parametro
#include <random>
const tint semilla = 38532164;
mt19937 gen(semilla);

tint mult(tint a, tint b, tint m)
{
	int largestBit = 0;
	while( (b >> largestBit) != 0)
		largestBit++;
	tint ans = 0;
	for(tint currentBit = largestBit - 1; currentBit >= 0; currentBit--)
	{
		ans = (ans + ans);
		if (ans >= m)
			ans -= m;

		if ( (b >> currentBit) & 1)
		{ 
			ans += a;
			if (ans >= m)
				ans -= m; 
		}
	}
	return ans;
}

bool esPrimoRM (tint n)
{
	if (n <= 1)
		return false;
	else if (n <= 3)
		return true;
	else if (n % 2 == 0)
		return false;
	else
	{
		uniform_int_distribution<tint> dis(2, n-2);
		tint kOrig = 0, m = n-1;
		while (m % 2 == 0)
		{
			kOrig++;
			m /= 2;
		}
		bool esPrimo = true;
		vector<tint> testigos = {2,3,5,7,11,13,17,19,23,29,31,37};
		for (auto a : testigos)
		{
			if (a < n)
			{
				tint b = potLogMod(a,m,n), k = kOrig;
				if (b == 1 or b == n-1)
					continue;
				else
				{
					forn(j,k)
					{
						b = mult(b,b,n);
						if (b == n-1)
							break;
						else if (b == 1)
						{
							esPrimo = false;
							break;
						}
					}
					if (b != n-1)
					{
						esPrimo = false;
						break;
					}
				}
			}
		}
		return esPrimo;
	}
}
\end{code}

\subsection{Pollard-Rho}
\begin{code}
// USA: Rabin-Miller
tint gcd (tint a, tint b)
{
	if (a == 0)
		return b;
	return gcd (b % a, a);
}
void factorizar (tint n, map<tint,tint> &f)
{
	while (n > 1)
	{
		if (esPrimoRM(n))
		{
			f[n]++;
			n /= n;
		}
		else
		{
			uniform_int_distribution<tint> dis(1, n-1);
			tint a = dis(gen), b = dis(gen), x = 2, y = 2, d;
			do
			{
				x = (mult(x,x,n) + mult(a,x,n) + b) % n;
				y = (mult(y,y,n) + mult(a,y,n) + b) % n;
				y = (mult(y,y,n) + mult(a,y,n) + b) % n;
				d = gcd(abs(x-y),n);
			}
			while (d == 1);
			if (d != n)
			{
				factorizar(d,f);
				n /= d;
			}
				
		}
	}
}

\end{code}

\subsection{FFT}
\begin{code}
// USA : "PotLog" e "InvMod" con nmod = mod 
const tint mod = (1 << 21)*11 + 1 ; // es re primo
const tint root = 38;
const tint root_1 = 21247462;
const tint root_pw = 1 << 21 ; // largo del arreglo
/*
 * const tint mod = 7340033;
 * const tint root = 5 ;
 * const tint root_1 = 4404020 ;
 * const tint root_pw = 1 << 20 ; 
 */
 
tint modulo (tint n)
{
	return ((n % mod) + mod) % mod;
}
void fft (vector <tint> &a, bool invert )
{
	tint n = a. size();
	for (tint i = 1 , j = 0 ; i < n ; ++ i )
	{
		tint bit = n >> 1 ;
		while(j >= bit)
		{
			j -= bit ;
			bit >>= 1;
		}
		j += bit ;
		if ( i < j )
			swap (a[i],a[j]);
	}
	for (tint len = 2 ; len <= n ; len <<= 1) 
	{
		tint wlen = root;
		if (invert)
			wlen = root_1;
		for (tint i = len ; i < root_pw ; i <<= 1)
			wlen = modulo(wlen * wlen);
		for (tint i = 0 ; i < n ; i += len ) 
		{
			tint w = 1 ;
			forn(j,len/2)
			{
				tint u = a[i+j], v = modulo(a[i+j+len/2] * w) ;
				a[i+j] = modulo(u+v);
				a[i+j + len/2] = modulo(u - v);
				w = modulo(w * wlen) ;
			}
		}
	}
	
	if (invert) 
	{
		tint nrev = invMod(n);
		forn(i,n)
			a[i] = modulo(a[i] * nrev) ;
	}
}

void multiply (const vector<tint> &a, const vector<tint> &b, vector<tint> &res) 
{
	vector<tint> fa(a.begin(), a.end() ), fb(b.begin(), b.end() );
	tint n = 1 ;
	while (n < max(tint(a.size()), tint(b.size())))
		n <<= 1;
	n <<= 1 ;
	fa.resize(n), fb.resize(n);
	fft (fa, false) , fft(fb, false);
	forn(i,n)
		fa[i] *= fb[i];
	fft(fa, true);
	res = fa;
} 
// USO:
int main()
{
	vector<tint> a = {1,0,0,1};
	vector<tint> b = {1,0,0,1};
	vector<tint> res;
	multiply(a,b,res);
	for (auto x : res)
		cout << x << " " ; // 1 0 0 2 0 0 1 0 
	cout << endl;
	return 0;
}

\end{code}

\subsection{Regla de Simpson (Integracion Numerica)}
\begin{code}
// f (x) una funcion definidia
ldouble a, b; // extremos de integracion
const int N = 1000*1000; // cantidad de nodos en la malla
ldouble s = 0; // resultado de la integral
ldouble h = (b - a) / N;
forn(i,N)
{
	ldouble x = a + h * i;
	s += f(x) * ((i==0 || i==N) ? 1 : ((i&1)==0) ? 2 : 4);
}
s *= h / 3; // es el resultado de integrar f en (a,b)
\end{code}
