\section{Arboles}

\subsection{Union-Find (Guty)}
\begin{code}
const tint maxN = 131072;
vector<tint> caminito;
tint representante[maxN];
tint tamanho[maxN];

void inicializar (tint n) {
	forn(i,n) {
		representante[i] = i;
		tamanho[i] = 1;
	}
}

tint find (tint x) {
	caminito = {};
	while (x != representante[x]) {
		caminito.push_back(x);
		x = representante[x];	
	}
	for (auto z : caminito)
		representante[z] = x;
	return x;
}

bool same (tint a, tint b) { return (find(a) == find(b)); }

void unite (tint a, tint b) {
	a = find(a);
	b = find(b);
	if (tamanho[a] < tamanho[b])
		swap(a,b);
	tamanho[a] += tamanho[b];
	representante[b] = a;	
}
\end{code}

\subsection{Union-Find (Jonaz)}
\begin{code}
class UF {
private: vector<int> p, rank; int comps;
public: 
  UF(int N) {
    rank.assign(N, 0); comps = N;
    p.assign(N, 0); forn(i,N) p[i] = i;
  }
  int findSet(int i) { return (p[i] == i) ? i : (p[i] = findSet(p[i])); }
  bool sameSet(int i, int j) { return findSet(i) == findSet(j); }
  void unionSet(int i, int j) {
    if (!sameSet(i,j)) {
      int x = findSet(i), y = findSet(j);
      if (rank[x] > rank[y]) p[y] = x;
      else { 
        p[x] = y; 
        if (rank[x] == rank[y]) rank[y]++;
      }
      comps--;
    }
  }
  int components() { return comps; }
};
\end{code}

\subsection{Kruskal (usa UF de Jonaz)}
\begin{code}
struct Arista {
  tint peso, start, end;
  Arista(tint s, tint e, tint p) : peso(p), start(s), end(e) {}
  bool operator<(const Arista& o) const {
    return make_tuple(peso, start, end) < make_tuple(o.peso, o.start, o.end);
}};
// Devuelve el peso del AGM, y en `agm' deja las aristas del mismo. 
tint kruskal(vector<Arista> &ars, tint size, vector<Arista> &agm) {
  sort(ars.begin(), ars.end());
  tint min_peso = 0;
  UF uf(size);
  for(auto &a : ars) {
    if (!uf.sameSet(a.start, a.end)) {
      min_peso += a.peso;
      uf.unionSet(a.start, a.end);
      agm.push_back(a);
      if ((tint)agm.size() == size-1) break; // Esto es que ya tiene V-1 aristas
  }}
  return min_peso;
}
\end{code}

%~ \subsection{LCA - Segment Tree (Jonaz)}
%~ \begin{code}
%~ \end{code}

\subsection{Binary Lifting (saltitos potencia de 2)}
\begin{code}
const tint maxN = 32768; // cantidad de nodos
const tint maxK = 16;	 // lg(cantidadDeNodos)
const tint NEUTRO = 1e8; // neutro de la operacion (ejemplo: minimo)

tint d[maxN]; // profundidad
pair<tint,tint> p[maxN][maxK]; 
// {ancestro a distancia 2^k, Lo que queremos entre los 2^k ancestros}

void dfs(tint actual, vector<vector<pair<tint,tint> > > &ladj, tint padre) {
	d[actual] = d[padre]+1;
	for (auto x : ladj[actual])
		if (x.first != padre) {
			p[x.first][0] = {actual,x.second};
			dfs(x.first,ladj,actual);
		}
}

tint subir(tint a, tint c, tint &ans, bool tomaMinimo) {
	tint k = 0;
	while (c > 0) {
		if (c % 2) {
			if (tomaMinimo)
				ans = min(ans,p[a][k].second);
			a = p[a][k].first;
		}
		k++;
		c /= 2;
	}
	return a;
}

tint answer (tint a, tint b) {
	// IGUALAMOS LAS ALTURAS
	if (d[a] < d[b])
		swap(a,b);
	tint w = d[a] - d[b], ans = NEUTRO;
	a = subir(a,w,ans,true);
	
	// HACEMOS LA BINARY PARA BUSCAR EL LCA
	tint cInf = 0, cSup = maxN; 
	while (cSup - cInf > 1) {
		tint ra = a, rb = b;
		tint c = (cSup+cInf)/2;
		ra = subir(ra,c,ans,false);
		rb = subir(rb,c,ans,false);
		if (ra == rb)
			cSup = c;
		else
			cInf = c;
	}
	// SUBIMOS LO QUE HAGA FALTA PARA LLEGAR AL LCA
	cSup *= (a != b);
	a = subir(a,cSup,ans,true);
	b = subir(b,cSup,ans,true);
	return ans;
}

// INICIALIZACION
int main() {
	forn(i,maxN)
	forn(k,maxK)
		p[i][k] = {-1,NEUTRO};
	// HACEMOS EL PRIMER PASO EN FUNCION DEL GRAFO
	vector<vector<pair<tint,tint> > > ladj (maxN); // listaDeAdyacencia del arbol
	d[0] = -1;
	dfs(0,ladj,0);
	// LLENADO DE LA TABLA
	forsn(k,1,maxK)	
	forn(i,maxN) {
		tint ancestro = p[i][k-1].first;
		if (ancestro >= 0)
			p[i][k] = {p[ancestro][k-1].first, 
					       min(p[i][k-1].second,p[ancestro][k-1].second) };
	}
}
\end{code}
