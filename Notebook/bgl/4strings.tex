\section{Strings}

\subsection{ToString/ToNumber}
\begin{code}
#include <iostream>
#include <string>
#include <sstream>

tint toNumber (string s)
{
	tint Number;
	if ( ! (istringstream(s) >> Number) )
		Number = 0; // el string vacio lo manda al cero
	return Number;
}

string toString (tint number)
{    
    ostringstream ostr;
    ostr << number;
    return  ostr.str();
}
\end{code}

\subsection{Tablita de Bordes}
\begin{code}
// Complejidad O(N)
// Devuelve arreglo, en la posicion i la longitud del maximo borde hasta st[i]. 
// En el ejemplo "abracadabra" devuelve 0 0 0 1 0 1 0 1 2 3 4 
vector<int> calcularBordes(string st) {
	int i=1, j=0, n=st.size();
	vector<int> bordes(n, 0);
	while(i<n) {
		while(j>0 && st[i] != st[j])
			j = bordes[j-1];
		if (st[i] == st[j])
			j++;
		bordes[i++] = j;
	}
	return bordes;
}
// String Matching con Bordes: 
// Concatenar S+'$'+T (donde $ no aparece en S ni T) 
// S chico, T grande
// Calcular bordes, siempre que el borde maximo para alguna 
// posicion correspondiente a T sea de longitud |S| 
// entonces hay un substring en T que coincide con S. 
\end{code}

\subsection{Knuth-Morris-Pratt (KMP) (Jonaz)}
\begin{code}
\end{code}

\subsection{Subsecuencia Comun mas larga}
\begin{code}
const tint maxN = 128;

tint p[maxN][maxN];
// Llamar lcs(s1,s2,n,m)
tint lcs(string &s1, string &s2,tint i,tint j)
{
	if (p[i][j] == -1)
	{
		if (i == 0 or j == 0)
		p[i][j] = 0;
		else
		{
			if (s1[i-1] == s2[j-1])
			p[i][j] = 1 + lcs(s1,s2,i-1,j-1);
			else
			{
				p[i][j] = max(p[i][j],lcs(s1,s2,i-1,j));
				p[i][j] = max(p[i][j],lcs(s1,s2,i,j-1));
			}
		}
	}
	return p[i][j];
}
\end{code}

\subsection{Edit-Distance}
\begin{code}
 // Minima distancia entre strings si lo que se puede es: INSERTAR, REMOVER, MODIFICAR, SWAPS ADYACENTES
const tint maxN = 1024; // maximo largo de los strings
const tint INFINITO = 1e15;
string s1,s2;
tint dist[maxN][maxN];

tint f(tint i, tint j)
{
	// Si un string es vacio, hay que borrar todo el otro
	if (i == -1 or j == -1) 
		return max(i,j)+1;
	if (dist[i][j] == INFINITO)
	{
		tint mini = INFINITO;		
     	// Lo mejor de borrar el i-esimo de s1 o insertar al final de s1 a s2[j]
		mini = min(mini,min(f(i-1,j)+1,f(i,j-1)+1));
		if (s1[i] == s2[j]) // Si coinciden, dejo como esta y resuelvo lo  anterior
			mini = min(mini,f(i-1,j-1)); 
		else // Modificar s1[i] a s2[j] y resolver lo anterior
			mini = min(mini,f(i-1,j-1)+1); 
			
		// Borramos los intermedios y swapeamos los ultimos 2 si funciona, lo hago y resuelvo lo anterior	
		forn(k,i) 
		{
			if (i >= 1 && j >= 1 && s1[i] == s2[j-1] && s1[i-k-1] == s2[j])
				mini = min(mini,f(i-k-2,j-2)+k+1);
		}
		dist[i][j] = mini;
	}
	return dist[i][j];
}
// USO: 
int main()
{
	tint n = s1.size(), m = s2.size();
	forn(i,n)
	forn(j,m)
		dist[i][j] = INFINITO;
	cout << f(n-1,m-1) << "\n";	
	return 0;
}
\end{code}

\subsection{Substring Palindromo (esPalindromo(s[i..j]))}
\begin{code}
// Asumo i < j
bool esPalindromo (tint i, tint j, vector<vector<tint> > &r, tint n)
{
	if (i+j >= n)
		return (r[i+j][n-i] - r[i+j][n-j-1]) == j-i+1;
	else
		return (r[i+j][j+1] - r[i+j][i]) == j-i+1;
}
// USO:
int main()
{
		tint n = s.size(); // s nuestro string
		vector<vector<tint> > v (n, vector<tint> (n,0));
		forn(i,n)
		forn(j,n)
			v[i][j] = (s[i] == s[j]);
		vector<vector<tint> > r (2*n-1,vector<tint> (n+1,0));
		
		forn(i,2*n-1)
		{
			tint sum = 0, x = min(i,n-1), y = 0;
			if (i >= n)
				y = i-n+1;
			forn(j,n)
			{
				if (x >=0 && y < n)
					sum += v[x--][y++];
				r[i][j+1] = sum;
			}
		}
		// Ahora podemos preguntar si es palindromo s[i..j]
}
\end{code}

