\section{Varios}

\subsection{Longest Incresing Subsequence (LIS)}
\begin{code}
tint LIS(vector<tint> &v) {
  if (v.empty()) return 0;
  
  tint l = 0;             // ultimo lugar de tails hasta ahora
  vi tails(v.size(), 0);  // candidatos de final de sub secuencias
  tails[l] = v[0];
  
  forsn(i,1,v.size()) {
    // con upper_bound es no-decreciente
    tint me = lower_bound(tails.begin(),tails.begin()+l+1, v[i])-tails.begin();
    tails[me] = v[i];
    if (me > l) l = me;
  }
  return l + 1;
}
\end{code}

\subsection{Maximum Subarray Sum}
\begin{code}
tint maximumSum (vector<tint> &a) // a no vacio
{
	tint maxTotal = a[0], maxAca = a[0], n = a.size();
	forsn(i,1,n)
	{
		maxAca = max(a[i],maxAca + a[i]);
		maxTotal = max(maxTotal,maxAca);
	}
	return maxTotal;
}
\end{code}

\subsection{Rotar 90° una matriz (sentido horario)}
\begin{code}
void rotar (vector<string> &origi) 
{
	tint n = origi.size();
	string aux (n,'x');
	vector<string> rotado (n,aux);
	forn(i,n)
	forn(j,n)
		rotado[j][n-i-1] = origi[i][j];
	origi = rotado;
}
\end{code}

\subsection{Random + Imprimir Doubles}
\begin{code}
#include <iostream>
#include <random>
#include <iomanip>

using namespace std;

random_device rd;
mt19937 gen(rd());
uniform_int_distribution<int> dis1(1, 10000);
uniform_real_distribution<long double> dis2(1, 10000);

int main()
{
	cout <<  dis1(gen) << "\n";
	cout <<  fixed << showpoint << setprecision(16) << dis2(gen) << "\n";
	return 0;
}
\end{code}

\subsection{Slding Window RMQ}
\begin{code}
void agrandarVentana (tint &r, deque<pair<tint,tint> > &rmq, vector<tint> &v)
{
	while (!rmq.empty() && rmq.back().first >= v[r])
		rmq.pop_back();
	rmq.push_back({v[r],r});
	r++;
	
}

void achicarVentana (tint &l, deque<pair<tint,tint> > &rmq)
{
	if (l == rmq.front().second)
		rmq.pop_front();
	l++;
}

pair<tint,tint> minimoVentana (deque<pair<tint,tint> > &rmq)
{
	return rmq.front();
}
// USO: En todo momento tenemos el minimo entre [l,r)
int main()
{
	deque<pair<tint,tint> > rmq; // {numero,indice}
	tint l = 0, r = 0;//  l   . r
	vector<tint> v = {1,2,3,4,3,2,2,3,4};
	agrandarVentana(r,rmq,v);
	agrandarVentana(r,rmq,v);
	agrandarVentana(r,rmq,v);
	agrandarVentana(r,rmq,v);
	agrandarVentana(r,rmq,v);
	achicarVentana(l,rmq);
	achicarVentana(l,rmq);
	cout << minimoVentana(rmq).first << endl;	// {3,4}
	return 0;
}
\end{code}

\subsection{Ternary Search}
\begin{code}
// Ternary en ENTEROS
tint miniTernarySearch (tint a, tint b) // En [a,b] esta el minimo
{
	tint l = a, r = b;
	while (abs(r - l) > 5)
	{
		tint al = (2*l + r)/3;
		tint br = (l + 2*r)/3;
		if (f(al) > f(br)) // cambiar a "<" para maximo
			l = al;
		else
			r = br;			
	}
	tint ans = 1e16;
	forsn(k,l,r+1)
		ans = min(ans,f(k));  // cambiar por "max" para maximo
	return ans;
}
//Ternary en FLOATING POINT
ldouble miniTernarySearch (ldouble tL, ldouble tR) // En [tL, tR] esta el minimo
{
	while (abs(tR - tL) > epsilon)
	{
		ldouble tLThird = (2.0*tL + tR)/3.0;
		ldouble tRThird = (tL + 2.0*tR)/3.0;
		if (f(tLeftThird) > f(tRightThird)) // cambiar a "<" para maximo
			tLeft = tLeftThird;
		else
			tRight = tRightThird;			
	}
	return f((tLeft+tRight)/2.0);
}
\end{code}
