\section{Geometria}

\subsection{Tipo PUNTO y Operaciones}
\begin{code}
const ldouble epsilon = 1e-10;
const ldouble pi = acos(-1);

struct Punto
{
	ldouble x,y;
	Punto (ldouble xx, ldouble yy)
	{
		x = xx;
		y = yy;
	}
	Punto()
	{
		x = 0.0;
		y = 0.0;
	}
};
Punto operator + (Punto p1, Punto p2)
{
	return Punto(p1.x+p2.x,p1.y+p2.y);
}
Punto operator - (Punto p1, Punto p2)
{
	return Punto(p1.x-p2.x,p1.y-p2.y);
}
Punto operator * (ldouble lambda, Punto p)
{
	return Punto(lambda*p.x, lambda*p.y);
}
ldouble operator * (Punto p1, Punto p2)
{
	return p1.x*p2.x+p1.y*p2.y;
}
ldouble operator ^ (Punto p1, Punto p2)
{
	return p1.x*p2.y - p1.y*p2.x;
}
Punto operator ~ (Punto p)
{
	return Punto(-p.y,p.x);
}
ldouble norma (Punto p)
{
	return sqrt(p.x*p.x+p.y*p.y);
}
bool operator < (Punto p1, Punto p2)
{
	return make_pair(p1.x,p1.y) < make_pair(p2.x,p2.y);
}
bool operator == (Punto p1, Punto p2)
{
	return ((abs(p1.x-p2.x) < epsilon) && (abs(p1.y-p2.y) < epsilon));
}
\end{code}

\subsection{Area de Poligono}
\begin{code}
ldouble areaTriangulo (Punto p1, Punto p2, Punto p3)
{
	return abs((p1-p3)^(p1-p2))/2.0;
}

ldouble areaPoligono(vector<Punto> &polygon)
{
	ldouble area = 0.0;
	tint n = polygon.size();
	forn(i,n)
		area += polygon[i]^polygon[(i+1)%n];
	return abs(area)/2.0;
}
\end{code}

\subsection{Punto en Poligono}
\begin{code}
bool adentroPoligono(vector<Punto> &polygon, Punto p) // polygon EN EL SENTIDO DE LAS AGUJAS
{
	bool adentro = true;
	tint n = polygon.size();
	forn(i,n)
		adentro &= (((p-polygon[i])^(p-polygon[(i+1)%n])) < 0);
	return adentro;
}
\end{code}

\subsection{Interseccion de Segmentos}
\begin{code}
struct Segmento
{
	Punto start,end,dir;
	Segmento (Punto ss, Punto ee)
	{
		start = ss;
		end = ee;
		dir = ee-ss;
	}
};
// res.second == 0 -> NO HAY INTERSECCION
// res.second == 1 -> INTERSECAN EN UN PUNTO (que esta en res.first)
// res.second == 2 -> SON COLINEALES E INTERSECAN EN TODO UN SEGMENTO (Da un extremo, si queremos el otro, correr otra vez con "otroExtremo" = true)
pair<Punto,tint> interSeg (Segmento s1, Segmento s2, bool otroExtremo ) 
{
	if ((abs(s1.dir ^ s2.dir)) < epsilon) // son colineales
	{
		vector<pair<Punto,tint> > aux = {{s1.start - epsilon*s1.dir,1},
																	 	 {s1.end   + epsilon*s1.dir,1},
															 			 {s2.start - epsilon*s2.dir,2},
															 			 {s2.end   + epsilon*s2.dir,2}};
		sort(aux.begin(),aux.end());
		if (aux[0].second != aux[1].second)
			return make_pair(aux[1+otroExtremo].first,2);
		else
			return make_pair(Punto(),0);
	}
	else
	{
		ldouble alfa = ((s2.start-s1.start)^s2.dir) / (s1.dir^s2.dir);
		if (0 <= alfa && alfa <= 1)
			return make_pair(s1.start+alfa*s1.dir,1);
		else
			return make_pair(Punto(),0);
	}
}
\end{code}

\subsection{Angulo Entre Puntos y Distancia entre Segmentos}
\begin{code}
ldouble angEntre (Punto p1, Punto p2, Punto p3) // P1^P2P3
{
	ldouble a = norma(p2-p3);
	ldouble b = norma(p1-p3);
	ldouble c = norma(p2-p1);
	return acos((a*a+c*c-b*b)/(2*a*c));
}

ldouble dPuntoSeg (Punto p, Segmento s)
{
	if (angEntre(p,s.start,s.end) > pi/2 or angEntre(p,s.end,s.start) > pi/2)
		return min(norma(p-s.start),norma(p-(s.end)));
	else
		return abs( ((s.start-p)^(s.end-p)) / (norma(s.dir)) );
}

ldouble dEntreSeg(Segmento s1, Segmento s2)
{
	ldouble a = min(dPuntoSeg(s1.start,s2),dPuntoSeg(s1.end,s2));
	ldouble b = min(dPuntoSeg(s2.start,s1),dPuntoSeg(s2.end,s1));
	return (interSeg(s1,s2,false).second == 0) * min(a,b);
}
\end{code}

\subsection{Convex-Hull (2D) (Jonaz)}
\begin{code}
\end{code}

\subsection{Sweep Line Facil (Interseccion de Segmentos/Closest Pair)}
\begin{code}
\end{code}

\subsection{Sweep Line Dificil (Union de Rectángulos)}
\begin{code}
\end{code}

\subsection{Radial Sweep}
\begin{code}
\end{code}

\subsection{Minimum Bounding Circle}
\begin{code}
\end{code}
