\section{Grafos}

\subsection{Dijsktra}
\begin{code}
const tint INFINITO = 1e15;

// parent : Inicializar (n,{}) : Guarda donde se realiza la minima distancia
// ladj : Por cada vertice, un par {indice,peso}

void dijkstra (tint comienzo, vector<vector<pair<tint,tint> > > &ladj, 
vector<tint> &distance, vector<vector<tint> > &parent)
{
	priority_queue <pair<tint,tint> > q; // {-peso,indice}
	tint n = distance.size();
	forn(i,n)
		distance[i] = (i != comienzo)*INFINITO;
	vector<tint> procesado (n,0);
	q.push({0,comienzo});
	while (!q.empty())
	{
		tint actual = q.top().second;
		q.pop();
		if (!procesado[actual])
		{
			procesado[actual] = 1;
			for (auto vecino : ladj[actual])
			{
				if (distance[actual] + vecino.second < distance[vecino.first])
				{
					distance[vecino.first] = distance[actual] + vecino.second;
					q.push({-distance[vecino.first],vecino.first});
					parent[vecino.first] = {actual};
				}
				else if (distance[actual] + vecino.second == distance[vecino.first])
					parent[vecino.first].push_back(actual);
			}
		}
	}
}
// En distance quedan las minimas distancias desde comienzo
\end{code}

\subsection{TopoSort y Kosaraju (Jonaz)}
\begin{code}
typedef vector<tint> vi;
void dfsTopo(vector<vi> &g, tint s, vi &vis, vi &ord, vi &comp) {
	vis[s] = true;
	for(auto ad : g[s]) if (!vis[ad]) dfsTopo(g, ad, vis, ord, comp);
	ord.push_back(s);
	comp.push_back(s);
}
vi topoSort(vector<vi> &g) { // Devuelve el orden topologico
  int N = g.size();
  vi vis, ord, aux;
	vis.assign(N, 0);
	forn(i,N) if (!vis[i]) dfsTopo(g, i, vis, ord, aux);
	reverse(ord.begin(), ord.end());
  return ord;
}
// Devuelve las componentes en orden topologico
vector<vi> kosaraju(vector<vi> &graf) { 
	vi ord = topoSort(graf);
  // Invertimos el grafo
  tint N = graf.size();
  vector<vi> grafInv(N, vi());
  forn(i,N) for(auto j : graf[i]) grafInv[j].push_back(i);
  
  vi vis(N, false), aux;
  vector<vi> comps;
	for (auto o : ord)
  if (!vis[o]) {
    vi comp; dfsTopo(grafInv, o, vis, aux, comp);
    comps.push_back(comp);
  }
	return comps;
}
\end{code}

\subsection{2-SAT (Jonaz)}
\begin{code}
\end{code}

\subsection{Puentes, Puntos de Articulacion y Biconexas (Jonaz)}
\begin{code}
\end{code}

\subsection{SPFA}
\begin{code}
const tint maxN = 16384; // cantidad de nodos
const tint INFINITO = 1e15; // suma de modulos de las aristas o algo asi

tint best[maxN];
bool adentro[maxN];
// ladj : {indice,peso}
void spfa (tint start, vector<vector<pair<tint,tint> > > &ladj)
{
	tint n = ladj.size();
	forn(i,n)
		best[i] = (i != start)*INFINITO;
	vector<tint> vecinos = {start}, nuevosVecinos;
	while (!vecinos.empty())
	{
		tint actual = vecinos.back();
		vecinos.pop_back();
		adentro[actual] = false;
		for (auto vecino : ladj[actual])
		{
			if (best[actual] + vecino.second < best[vecino.first])
			{
				best[vecino.first] = best[actual] + vecino.second;
				if (!adentro[vecino.first])
				{
					nuevosVecinos.push_back(vecino.first);
					adentro[vecino.first] = 1;
				}
			}
		}
		if (vecinos.empty())
			vecinos.swap(nuevosVecinos);
	}
}
\end{code}

\subsection{Ciclo Hamiltoniano Minimo}
\begin{code}
const tint INFINITO = 1e15;

tint minimumHamiltonianCycle (vector<vector<tint> > &d)
{
	tint r = d.size(), minHam = INFINITO;
	if (r > 1)
	{
		vector<vector<tint> > dp ((1 << r), vector<tint> (r,INFINITO));
		dp[1][0] = 0;
		for(tint mask = 1; mask < (1 << r); mask += 2)
		forn(i,r)
			if ( (i > 0) && (mask & (1 << i)) && (mask & 1) )
				forn(j,r)
					if ((i != j) && (mask & (1 << j)))
						dp[mask][i] = min(dp[mask][i],dp[mask ^ (1 << i)][j] + d[j][i]);
		
		forsn(i,1,r)
			minHam = min(minHam,dp[(1 << r) - 1][i] + d[i][0]);
	}
	else
		minHam = d[0][0];
	return minHam;
}
\end{code}

\subsection{Dinic (aguanta multiejes y autoejes)}
\begin{code}
const tint maxN = 512;
const tint INFINITO = 1e15;
struct Arista
{
	tint start,end,capacity,flow;
	Arista (tint ss, tint ee, tint cc, tint ff)
	{
		start = ss;
		end = ee;
		capacity = cc;
		flow = ff;
	}
};

vector<Arista>  red; // Red residual
vector<tint> ladj [maxN]; // (guarda vecinos como indices en red)

tint n, s, t; // #Nodos, source, sink
tint ultimoVecino [maxN]; // ultimo vecino visitado en dfs
tint nivel [maxN]; // Nivel del bfs

void agregarArista (tint ss, tint ee, tint c)
{
	ladj[ss].push_back( tint (red.size())); // guardamos el indice
	red.push_back(Arista(ss,ee,c,0));
	ladj[ee].push_back( tint (red.size()));
	red.push_back(Arista(ee,ss,c,0));
}


bool bfs ()
{
	forn(i,n+1)
		nivel[i] = -1;
	vector<tint> vecinos = {s}, nuevosVecinos;
	nivel[s] = 0;
	while (!vecinos.empty() && nivel[t] == -1)
	{
		tint actual = vecinos.back();
		vecinos.pop_back();
		for (auto iArista : ladj[actual])
		{
			tint vecino = red[iArista].end;
			// Si bajo en uno el nivel y puedo mandar flujo en la red residual
			if (nivel[vecino] == -1 && red[iArista].flow < red[iArista].capacity) 
			{
				nivel[vecino] = nivel[actual] + 1;
				nuevosVecinos.push_back(vecino);
			}
		}
		if (vecinos.empty())
		{
			swap(vecinos,nuevosVecinos);
			nuevosVecinos = {};
		}
	}
	return (nivel[t] != -1);
}

tint dfs (tint actual, tint flujo)
{
	if (flujo <= 0)
		return 0;
	else if (actual == t)
		return flujo;
	else
	{
		while (ultimoVecino[actual] < tint(ladj[actual].size()))
		{
			tint id = ladj[actual][ultimoVecino[actual]];
			if (nivel[red[id].end] == nivel[actual] + 1)
			{
				tint pushed = dfs(red[id].end,min(flujo,red[id].capacity-red[id].flow));
				if (pushed > 0)
				{
					red[id].flow += pushed;
					red[id^1].flow -= pushed;
					return pushed;
				}
			}
			ultimoVecino[actual]++;
		}
		return 0;	
	}
}

tint dinic ()
{
	tint flujo = 0;
	while (bfs())
	{
		
		forn(i,n+1)
			ultimoVecino[i] = 0;
		tint pushed = dfs(s,INFINITO);
		
		while (pushed > 0)
		{
			flujo += pushed;
			pushed = dfs(s,INFINITO);
		}
	}
	return flujo;
}
\end{code}

\subsection{Flujo de Costo Mínimo}
\begin{code}
\end{code}
