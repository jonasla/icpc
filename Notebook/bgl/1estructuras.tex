\section*{Cosas a tener en cuenta}
\subsection*{Flags de Compilaci\'on}
\begin{code}
g++ -std=c++11 -DACMTUYO -O2 -Wshadow -Wextra -D_GLIBCXX_DEBUG -Wall -c "%f"
g++ -std=c++11 -DACMTUYO -O2 -Wshadow -Wall -Wextra -D_GLIBCXX_DEBUG -o"%e" "%f"
time "./%e"
\end{code}

\section{Estructuras}

\subsection{Fenwick Tree}
\begin{code}
// TRABAJAR CON UN VECTOR INDEXADO EN 1 EN "fenwick" (DE TAMANO N+1)
void add (tint k, tint x, vector<tint> &fenwick) { // Suma x al indice k 
	tint n = fenwick.size() -1;
	while (k <= n) {
		fenwick[k] += x;
		k += (k & -k);
	}
}
// Devuelve la suma en el rango [1..k] (inclusive)
tint sum (tint k, vector<tint> &fenwick) {
	tint s = 0;
	while (k >= 1) {
		s += fenwick[k];
		k -= (k & -k);
	}
	return s;
}
\end{code}

\subsection{Trie}
\begin{code}
const int MAXN = 60000;

struct TrieNode {
  map<char, int> sig; 
  bool final = false;
  void reset() { sig.clear(); final = false; }
};

TrieNode trie[MAXN];
int trie_n = 1;

void resetTrie() {
  trie_n = 1;
  trie[0].reset();
}

void insertar(string st) {
  int pos = 0;
  for(int i=0; i<(int)st.size(); i++) {
    if (trie[pos].sig.find(st[i]) == trie[pos].sig.end()) {
      trie[pos].sig[st[i]] = trie_n;
      trie[trie_n].reset();
      trie_n++;
    }
    pos = trie[pos].sig[st[i]];
  }
  trie[pos].final = true;
}

bool buscar(string st) {
  int pos = 0;
  for(int i=0; i<(int)st.size(); i++) {
    if (trie[pos].sig.find(st[i]) == trie[pos].sig.end())
      return false;
    pos = trie[pos].sig[st[i]];
  }
  return (trie[pos].final == true);
}
\end{code}

\subsection{Segment Tree}
\begin{code}
// Nodo del segment tree
struct Nodo {
	tint x;
	Nodo (tint xx) { x = xx; }
};
// Operacion del segment tree : tiene que ser ASOCIATIVA
Nodo op (Nodo n1, Nodo n2) {
	return Nodo(n1.x+n2.x);
}
vector<Nodo> buildSegTree (vector<Nodo> &v ) {
	// Completa el tamanho
	tint k = 4, n = v.size();
	while (k < 2*n)
		k <<= 1;
	// Rellena las hojas
	vector<Nodo> seg (k, Nodo(0));
	forn(i,n)
		seg[(k >> 1)+i] = v[i];
	// Completa los padres
	while (k > 0) {
		seg[(k-1) >> 1] = op(seg[k-1],seg[k-2]);
		k -= 2;
	}	
	return seg; 
}
// i es el indice de [0,n) en el arreglo original
// Nodo es lo que queremos poner ahora como hoja
void update(tint i, Nodo nodo,vector<Nodo> &seg) {
	tint k = seg.size()/2 + i;
	seg[k] = nodo;
	while (k > 0) {
		seg[k >> 1] = op(seg[k],seg[k^1]);
		k >>= 1;
	}
}
Nodo queryAux(tint k, tint l, tint r, tint i, tint j, vector<Nodo> &seg) {	
	if (i <= l && r <= j)
		return seg[k];
	if (r <= i or l >= j)
		return Nodo(0); // Aca va el NEUTRO de la funcion "op"
	Nodo a = queryAux(2*k,l,(l+r) >> 1,i,j,seg);
	Nodo b = queryAux(2*k+1,(l+r) >> 1,r,i,j,seg);
	return op(a,b);	
}
// i,j son los indices del arreglo del que se hace la query
// la query se hace en [i,j)
Nodo query(tint i, tint j, vector<Nodo> &seg) {
	return queryAux(1,0,seg.size() >> 1,i,j,seg);
}
// USO:
int main() {
	tint n = 15;
	vector<Nodo> v (n, Nodo(0));
	forn(i,n)
		v[i] = Nodo((3*(i+1))% 7 - 9*(i-4)%13);
	vector<Nodo> seg = buildSegTree(v);
	forn(i,n)
		cout << v[i].x << " "; // 13 7 7 14 1 -5 -5 2 -4 -4 3 -10 -3 -3 -9 
	cout << endl;
	cout << query(3,11,seg).x << "\n"; // Devuelve 2
	update(6,Nodo(0),seg);
	cout << query(3,11,seg).x << "\n"; // Devuelve 7
	return 0;
}
\end{code}
