% LaTeXar con :
%  pdflatex notebook.tex
% o bien,
%  latex notebook.dvi
%  dvipdfm notebook.dvi
%
%	" The PDF file may contain up to 25 pages of reference material, single-sided,
%   letter or A4 size, with text and illustrations readable by a person with
%   correctable eyesight without magnification from a distance of 1/2 meter. "
%
\documentclass[10pt,landscape,twocolumn,a4paper,notitlepage]{article}
\usepackage{hyperref}
\usepackage[spanish, activeacute]{babel}
\usepackage{fancyhdr}
\usepackage{lastpage}
\usepackage{listings}
\usepackage{amssymb}
\usepackage{svnkw}
\usepackage{rotating}

%\svnidlong{$HeadURL: http://tpsh.exp.dc.uba.ar/svn/adeymo-acm/notebook/notebook.tex $}{$LastChangedDate: 2011-05-22 22:49:20 -0300 (dom, 22 may 2011) $}{$LastChangedRevision: 1 $}{$LastChangedBy: jonaz $}
\def\revision{\svnrev}

%%% M�rgenes
\setlength{\columnsep}{0.25in}    % default=10pt
\setlength{\columnseprule}{0.5pt}    % default=0pt (no line)

\addtolength{\textheight}{2.35in}
\addtolength{\topmargin}{-0.9in}     % ~ -0.5 del incremento anterior

\addtolength{\textwidth}{1.1in}
\addtolength{\oddsidemargin}{-0.60in} % -0.5 del incremento anterior
\setlength{\headwidth}{\textwidth}
\addtolength{\headwidth}{0.1in}

\setlength{\headsep}{0.08in}
\setlength{\parskip}{0in}
\setlength{\headheight}{15pt}
\setlength{\parindent}{0mm}

%%% Encabezado y pie de p�gina
\pagestyle{fancy}
\fancyhead[LO]{\leftmark\ -\ \rightmark}
%\fancyhead[C]{\textbf{BGL-UBA}}
\fancyhead[RO]{\textbf{BGL-UBA} - P\'agina \thepage\ de \pageref{LastPage}\ \begin{rotate}{270}\hspace{.8em}\underline{\textbf{Universidad de Buenos Aires - FCEN -- BGL}\hspace{4.1in}\ }\end{rotate}}
\fancyfoot{}
\fancyfoot[RO]{\begin{rotate}{270}\hspace{-10em}\textbf{Page \thepage\ of \pageref{LastPage}}\end{rotate}}
\renewcommand{\headrulewidth}{0.4pt}
\renewcommand{\footrulewidth}{0.4pt}
\renewcommand{\footruleskip}{0.2in}

%%% Configuraci�n de Listings
\lstloadlanguages{C++}
\lstnewenvironment{code}
	{%\lstset{	numbers=none, frame=lines, basicstyle=\small\ttfamily, }%
	 \csname lst@SetFirstLabel\endcsname}
	{\csname lst@SaveFirstLabel\endcsname}
\lstset{% general command to set parameter(s)
	language=C++, basicstyle=\small\ttfamily, keywordstyle=\slshape,
	emph=[1]{tipo,usa}, emphstyle={[1]\sffamily\bfseries},
	morekeywords={tint,forn,forsn},
	basewidth={0.47em,0.40em},
	columns=fixed, fontadjust, resetmargins, xrightmargin=5pt, xleftmargin=15pt,
	flexiblecolumns=false, tabsize=2, breaklines,	breakatwhitespace=false, extendedchars=true,
	numbers=left, numberstyle=\tiny, stepnumber=1, numbersep=9pt,
	frame=l, framesep=3pt,
}

%%% Macros
\def\nbtitle#1{\begin{Large}\begin{center}\textbf{#1}\end{center}\end{Large}}
\def\nbsection#1{\section{#1}}
\def\nbsubsection#1{\subsection{#1}}
\def\nbcoment#1{\begin{small}\textbf{#1}\end{small}}

\newcommand{\comb}[2]{\left( \begin{array}{c} #1 \\ #2 \end{array}\right)}

\begin{document}

\tableofcontents\newpage

%%% El texto propiamente dicho
\nbtitle{BGL-UBA - Reference}
\section{Cosas}

\subsection{Segment Tree (Guty)}
%\nbcoment{}
\begin{code}
// Nodo del segment tree
struct Nodo{
	tint x;
	Nodo (tint xx)
		x = xx;
};
// Operacion del segment tree : tiene que ser ASOCIATIVA
Nodo op (Nodo n1, Nodo n2){
	return Nodo(n1.x+n2.x);
}
vector<Nodo> buildSegTree (vector<Nodo> &v ){
	// Completa el tamanho
	tint k = 4, n = v.size();
	while (k < 2*n)
		k *= 2;
	// Rellena las hojas
	vector<Nodo> seg (k, Nodo(0));
	forn(i,n)
		seg[k/2+i] = v[i];
	// Completa los padres
	while (k > 0)	{
		seg[(k-1)/2] = op(seg[k-1],seg[k-2]);
		k -= 2;
	}		
	return seg; 
}
// i es el indice de [0,n) en el arreglo original
// Nodo es lo que queremos poner ahora como hoja
void update(tint i, Nodo nodo,vector<Nodo> &seg){
	tint k = seg.size()/2 + i;
	seg[k] = nodo;
	while (k > 0){
		seg[k >> 1] = op(seg[k],seg[k^1]);
		k /= 2;
	}
}
Nodo queryAux(tint k, tint l, tint r, tint i, tint j, vector<Nodo> &seg){
	if (i <= l && r <= j)
		return seg[k];
	if (r <= i or l >= j)
		return 0; // Aca va el NEUTRO de la funcion "op"
	Nodo a = queryAux(2*k,l,(l+r)/2,i,j,seg);
	Nodo b = queryAux(2*k+1,(l+r)/2,r,i,j,seg);
	return op(a,b);
	
}
// i,j son los indices del arreglo del que se hace la query
// la query se hace en [i,j)
Nodo query(tint i, tint j, vector<Nodo> &seg){
	return queryAux(1,0,seg.size()/2,i,j,seg);
}
int main(){
	tint n = 15;
	vector<Nodo> v (n, Nodo(0));
	forn(i,n)
		v[i] = Nodo((3*(i+1))% 7 - 9*(i-4)%13);
	vector<Nodo> seg = buildSegTree(v);
	imprimirVector(v);
	cout << query(3,11,seg).x << "\n";
	return 0;
}
\end{code}
\subsection{Inverso modular Primo y PotLog (Guty)}
\begin{code}
const tint nmod = 1000000007; // o el primo que deseamos

vector<tint> desBaseB (tint n, tint b){ // Calcula el desarrollo de n en base b
	if (n == 0)
		return {0};
	vector<tint> des;
	while (n > 0){
		des.push_back(n%b);
		n /= b;
	}
	reverse(des.begin(),des.end());
	return des;
}
tint potLogMod (tint x, tint y){ // Calcula: (x^y) mod nmod
	tint ans = 1;
	while (y > 0){
		if (y % 2)
			ans = (x * ans) % nmod;
		x = (x * x) % nmod;
		y /= 2;
	}
	return ans;
}
tint invLog(tint a){// Solo funciona si nmod es primo y devuelve un numero b tal que: (a*b) = 1 mod nmod 
	return potLogMod(a,nmod-2);
}
\end{code}
\subsection{Dijkstra (Guty)}
\begin{code}
const int INF = 1000000000; // Aca va una cota que funque para el problema en vez de 1000000000
struct Arista{
	tint v1,v2,peso;
	Arista(tint vv1, tint vv2, tint pp){
		v1 = vv1;
		v2 = vv2;
		peso = pp;
	}
};
bool operator < (Arista a1, Arista a2){
	return make_tuple(a1.peso,a1.v1,a1.v2) > make_tuple(a2.peso,a2.v1,a2.v2);
}
vector<tint> dijkstra (vector<vector<tint> > &ladj, vector<vector<tint> > &w, tint s){ // Devuelve un vector d, tal que d[v] es la minima distancia de "s" a "v"
	tint n = ladj.size();
	vector<tint> d (n,INF); 
	priority_queue<Arista> v;
	d[s] = 0;
	for(auto vecino : ladj[s])
		v.push(Arista(s,vecino,w[s][vecino]));
	while (!v.empty()){
		if (d[v.top().v2] == INF or d[v.top().v1] + v.top().peso <= d[v.top().v2]){
			Arista e = v.top();
			v.pop();
			d[e.v2] = d[e.v1] + e.peso;
			for(auto vecino : ladj[e.v2])
				v.push(Arista(e.v2,vecino,w[e.v2][vecino]));
		}else
			v.pop();
	}
	return d;
}
\end{code}
\subsection{Edmonds Karp (Guty)}
\begin{code}
// Hay que tener definidas de antemano:
// capacidades: Una matriz que en el lugar (i,j) guarda la capacidad que une al nodo i con j.
// ladj: Para cada nodo tiene la lista de vecinos. Notar que al comenzar el codigo se agregan las aristas que faltan para la red residual
// flow: Se debe dar un flujo inicial de 0
// flowPath: Guarda en el lugar (i,j) la cantidad de flujo que efectivamente pasa por la arista que une i con j. Inicialmente debe ser una matriz de ceros.

const tint INF = 999999999999;

tint maxFlow (vector<vector<tint> > &capacidades, vector<vector<tint> > &ladj, tint qNodos, tint source, tint terminal){
	tint flow = 0;
	vector<vector<tint> > flowPath (qNodos, vector<tint> (qNodos,0));
	tint capacityFound = -1;
	forn(i, ladj.size())
		for (auto &a : ladj[i])
			if (capacidades[i][a] != 0)
				ladj[a].push_back(i);
	while (capacityFound != 0){
		vector<tint> path (qNodos,-1);
		// Aca empieza el bfs
		
		path[source] = -2;
		capacityFound = 0;
		vector<tint> pathCapacity (qNodos,INF); // Aca va una cota para el flujo del problema
		deque<tint> visit = {source};
		while (!visit.empty()){
			tint actual = visit.front();
			visit.pop_front();
			for (auto vecino : ladj[actual]){
				if (capacidades[actual][vecino] > flowPath[actual][vecino] && path[vecino] == -1 ){
					path[vecino] = actual;
					pathCapacity[vecino] = min(pathCapacity[actual],capacidades[actual][vecino] - flowPath[actual][vecino]);
					if (vecino != terminal)
						visit.push_back(vecino);
					else{
						capacityFound = pathCapacity[vecino];
						visit.clear();
						break;
					}
				}
			}
		}
		// Aca termina el bfs
		if (capacityFound == 0)
			break;
		flow += capacityFound;
		tint v = terminal;
		while (v != source){
			tint u = path[v];
			flowPath[u][v] += capacityFound;
			flowPath[v][u] -= capacityFound;
			v = u;
		}
	}
	return flow;
}
\end{code}
\end{document}
