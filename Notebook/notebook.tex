% LaTeXar con :
%  pdflatex notebook.tex
% o bien,
%  latex notebook.dvi
%  dvipdfm notebook.dvi
%
%	" The PDF file may contain up to 25 pages of reference material, single-sided,
%   letter or A4 size, with text and illustrations readable by a person with
%   correctable eyesight without magnification from a distance of 1/2 meter. "
%
\documentclass[10pt,landscape,twocolumn,a4paper,notitlepage]{article}
\usepackage{hyperref}
\usepackage[spanish, activeacute]{babel}
\usepackage{fancyhdr}
\usepackage{lastpage}
\usepackage{listings}
\usepackage{amssymb}
\usepackage{svnkw}
\usepackage{rotating}

%\svnidlong{$HeadURL: http://tpsh.exp.dc.uba.ar/svn/adeymo-acm/notebook/notebook.tex $}{$LastChangedDate: 2011-05-22 22:49:20 -0300 (dom, 22 may 2011) $}{$LastChangedRevision: 1 $}{$LastChangedBy: jonaz $}
\def\revision{\svnrev}

%%% M�rgenes
\setlength{\columnsep}{0.25in}    % default=10pt
\setlength{\columnseprule}{0.5pt}    % default=0pt (no line)

\addtolength{\textheight}{2.35in}
\addtolength{\topmargin}{-0.9in}     % ~ -0.5 del incremento anterior

\addtolength{\textwidth}{1.1in}
\addtolength{\oddsidemargin}{-0.60in} % -0.5 del incremento anterior
\setlength{\headwidth}{\textwidth}
\addtolength{\headwidth}{0.1in}

\setlength{\headsep}{0.08in}
\setlength{\parskip}{0in}
\setlength{\headheight}{15pt}
\setlength{\parindent}{0mm}

%%% Encabezado y pie de p�gina
\pagestyle{fancy}
\fancyhead[LO]{\leftmark\ -\ \rightmark}
%\fancyhead[C]{\textbf{BGL-UBA}}
\fancyhead[RO]{\textbf{BGL-UBA} - P\'agina \thepage\ de \pageref{LastPage}\ \begin{rotate}{270}\hspace{.8em}\underline{\textbf{Universidad de Buenos Aires - FCEN -- BGL}\hspace{4.1in}\ }\end{rotate}}
\fancyfoot{}
\fancyfoot[RO]{\begin{rotate}{270}\hspace{-10em}\textbf{Page \thepage\ of \pageref{LastPage}}\end{rotate}}
\renewcommand{\headrulewidth}{0.4pt}
\renewcommand{\footrulewidth}{0.4pt}
\renewcommand{\footruleskip}{0.2in}

%%% Configuraci�n de Listings
\lstloadlanguages{C++}
\lstnewenvironment{code}
	{%\lstset{	numbers=none, frame=lines, basicstyle=\small\ttfamily, }%
	 \csname lst@SetFirstLabel\endcsname}
	{\csname lst@SaveFirstLabel\endcsname}
\lstset{% general command to set parameter(s)
	language=C++, basicstyle=\small\ttfamily, keywordstyle=\slshape,
	emph=[1]{tipo,usa}, emphstyle={[1]\sffamily\bfseries},
	morekeywords={tint,forn,forsn},
	basewidth={0.47em,0.40em},
	columns=fixed, fontadjust, resetmargins, xrightmargin=5pt, xleftmargin=15pt,
	flexiblecolumns=false, tabsize=2, breaklines,	breakatwhitespace=false, extendedchars=true,
	numbers=left, numberstyle=\tiny, stepnumber=1, numbersep=9pt,
	frame=l, framesep=3pt,
}

%%% Macros
\def\nbtitle#1{\begin{Large}\begin{center}\textbf{#1}\end{center}\end{Large}}
\def\nbsection#1{\section{#1}}
\def\nbsubsection#1{\subsection{#1}}
\def\nbcoment#1{\begin{small}\textbf{#1}\end{small}}

\newcommand{\comb}[2]{\left( \begin{array}{c} #1 \\ #2 \end{array}\right)}

\begin{document}

\tableofcontents\newpage

%%% El texto propiamente dicho
\nbtitle{BGL-UBA - Reference}

\section*{Cosas a tener en cuenta}

\subsection*{Flags de Compilaci�n}

\begin{code}
g++ -std=c++11 -DACMTUYO -O2 -Wshadow -Wextra -D_GLIBCXX_DEBUG -Wall -c "%f"
g++ -std=c++11 -DACMTUYO -O2 -Wshadow -Wall -Wextra -D_GLIBCXX_DEBUG -o"%e" "%f"
time "./%e"
\end{code}


\section{Estructuras}

\subsection{Fenwick Tree}
\begin{code}
// TRABAJAR CON UN VECTOR INDEXADO EN 1 EN "fenwick" (DE TAMANO N+1)
void add (tint k, tint x, vector<tint> &fenwick) // Suma x al indice k 
{
	tint n = fenwick.size() -1;
	while (k <= n)
	{
		fenwick[k] += x;
		k += (k & -k);
	}
}																																     (inclusive)
tint sum (tint k, vector<tint> &fenwick) // Devuelve la suma en el rango [1..k] 
{
	tint s = 0;
	while (k >= 1)
	{
		s += fenwick[k];
		k -= (k & -k);
	}
	return s;
}
\end{code}

\subsection{Trie (Jonaz)}

\begin{code}

\end{code}

\subsection{Segment Tree}

\begin{code}
// Nodo del segment tree
struct Nodo
{
	tint x;
	Nodo (tint xx)
	{
		x = xx;
	}
};
// Operacion del segment tree : tiene que ser ASOCIATIVA
Nodo op (Nodo n1, Nodo n2)
{
	return Nodo(n1.x+n2.x);
}
vector<Nodo> buildSegTree (vector<Nodo> &v )
{
	// Completa el tamanho
	tint k = 4, n = v.size();
	while (k < 2*n)
		k <<= 1;
	// Rellena las hojas
	vector<Nodo> seg (k, Nodo(0));
	forn(i,n)
		seg[(k >> 1)+i] = v[i];
	// Completa los padres
	while (k > 0)
	{
		seg[(k-1) >> 1] = op(seg[k-1],seg[k-2]);
		k -= 2;
	}	
	return seg; 
}
// i es el indice de [0,n) en el arreglo original
// Nodo es lo que queremos poner ahora como hoja
void update(tint i, Nodo nodo,vector<Nodo> &seg)
{
	tint k = seg.size()/2 + i;
	seg[k] = nodo;
	while (k > 0)
	{
		seg[k >> 1] = op(seg[k],seg[k^1]);
		k >>= 1;
	}
}
Nodo queryAux(tint k, tint l, tint r, tint i, tint j, vector<Nodo> &seg)
{	
	if (i <= l && r <= j)
		return seg[k];
	if (r <= i or l >= j)
		return Nodo(0); // Aca va el NEUTRO de la funcion "op"
	Nodo a = queryAux(2*k,l,(l+r) >> 1,i,j,seg);
	Nodo b = queryAux(2*k+1,(l+r) >> 1,r,i,j,seg);
	return op(a,b);	
}
// i,j son los indices del arreglo del que se hace la query
// la query se hace en [i,j)
Nodo query(tint i, tint j, vector<Nodo> &seg)
{
	return queryAux(1,0,seg.size() >> 1,i,j,seg);
}
// USO:
int main()
{
	tint n = 15;
	vector<Nodo> v (n, Nodo(0));
	forn(i,n)
		v[i] = Nodo((3*(i+1))% 7 - 9*(i-4)%13);
	vector<Nodo> seg = buildSegTree(v);
	forn(i,n)
		cout << v[i].x << " "; // 13 7 7 14 1 -5 -5 2 -4 -4 3 -10 -3 -3 -9 
	cout << endl;
	cout << query(3,11,seg).x << "\n"; // Devuelve 2
	update(6,Nodo(0),seg);
	cout << query(3,11,seg).x << "\n"; // Devuelve 7
	return 0;
}
\end{code}

\pagebreak

\section{Grafos}

\subsection{Dijsktra}

\begin{code}
const tint INFINITO = 1e15;

// parent : Inicializar (n,{}) : Guarda donde se realiza la minima distancia
// ladj : Por cada vertice, un par {indice,peso}

void dijkstra (tint comienzo, vector<vector<pair<tint,tint> > > &ladj, 
vector<tint> &distance, vector<vector<tint> > &parent)
{
	priority_queue <pair<tint,tint> > q; // {-peso,indice}
	tint n = distance.size();
	forn(i,n)
		distance[i] = (i != comienzo)*INFINITO;
	vector<tint> procesado (n,0);
	q.push({0,comienzo});
	while (!q.empty())
	{
		tint actual = q.top().second;
		q.pop();
		if (!procesado[actual])
		{
			procesado[actual] = 1;
			for (auto vecino : ladj[actual])
			{
				if (distance[actual] + vecino.second < distance[vecino.first])
				{
					distance[vecino.first] = distance[actual] + vecino.second;
					q.push({-distance[vecino.first],vecino.first});
					parent[vecino.first] = {actual};
				}
				else if (distance[actual] + vecino.second == distance[vecino.first])
					parent[vecino.first].push_back(actual);
			}
		}
	}
}
// En distance quedan las minimas distancias desde comienzo

\end{code}

\subsection{TopoSort (Jonaz)}

\begin{code}

\end{code}

\subsection{Kosaraju (Jonaz)}

\begin{code}

\end{code}

\subsection{2-SAT (Jonaz)}

\begin{code}

\end{code}

\subsection{Puentes, Puntos de Articulacion y Biconexas (Jonaz)}

\begin{code}

\end{code}

\subsection{SPFA}

\begin{code}
const tint maxN = 16384; // cantidad de nodos
const tint INFINITO = 1e15; // suma de modulos de las aristas o algo asi

tint best[maxN];
bool adentro[maxN];
// ladj : {indice,peso}
void spfa (tint start, vector<vector<pair<tint,tint> > > &ladj)
{
	tint n = ladj.size();
	forn(i,n)
		best[i] = (i != start)*INFINITO;
	vector<tint> vecinos = {start}, nuevosVecinos;
	while (!vecinos.empty())
	{
		tint actual = vecinos.back();
		vecinos.pop_back();
		adentro[actual] = false;
		for (auto vecino : ladj[actual])
		{
			if (best[actual] + vecino.second < best[vecino.first])
			{
				best[vecino.first] = best[actual] + vecino.second;
				if (!adentro[vecino.first])
				{
					nuevosVecinos.push_back(vecino.first);
					adentro[vecino.first] = 1;
				}
			}
		}
		if (vecinos.empty())
			vecinos.swap(nuevosVecinos);
	}
}
\end{code}

\subsection{Ciclo Hamiltoniano Minimo}
\begin{code}
const tint INFINITO = 1e15;

tint minimumHamiltonianCycle (vector<vector<tint> > &d)
{
	tint r = d.size(), minHam = INFINITO;
	if (r > 1)
	{
		vector<vector<tint> > dp ((1 << r), vector<tint> (r,INFINITO));
		dp[1][0] = 0;
		for(tint mask = 1; mask < (1 << r); mask += 2)
		forn(i,r)
			if ( (i > 0) && (mask & (1 << i)) && (mask & 1) )
				forn(j,r)
					if ((i != j) && (mask & (1 << j)))
						dp[mask][i] = min(dp[mask][i],dp[mask ^ (1 << i)][j] + d[j][i]);
		
		forsn(i,1,r)
			minHam = min(minHam,dp[(1 << r) - 1][i] + d[i][0]);
	}
	else
		minHam = d[0][0];
	return minHam;
}
\end{code}


\subsection{Dinic (aguanta multiejes y autoejes)}

\begin{code}

const tint maxN = 512;
const tint INFINITO = 1e15;
struct Arista
{
	tint start,end,capacity,flow;
	Arista (tint ss, tint ee, tint cc, tint ff)
	{
		start = ss;
		end = ee;
		capacity = cc;
		flow = ff;
	}
};

vector<Arista>  red; // Red residual
vector<tint> ladj [maxN]; // (guarda vecinos como indices en red)

tint n, s, t; // #Nodos, source, sink
tint ultimoVecino [maxN]; // ultimo vecino visitado en dfs
tint nivel [maxN]; // Nivel del bfs

void agregarArista (tint ss, tint ee, tint c)
{
	ladj[ss].push_back( tint (red.size())); // guardamos el indice
	red.push_back(Arista(ss,ee,c,0));
	ladj[ee].push_back( tint (red.size()));
	red.push_back(Arista(ee,ss,c,0));
}


bool bfs ()
{
	forn(i,n+1)
		nivel[i] = -1;
	vector<tint> vecinos = {s}, nuevosVecinos;
	nivel[s] = 0;
	while (!vecinos.empty() && nivel[t] == -1)
	{
		tint actual = vecinos.back();
		vecinos.pop_back();
		for (auto iArista : ladj[actual])
		{
			tint vecino = red[iArista].end;
			// Si bajo en uno el nivel y puedo mandar flujo en la red residual
			if (nivel[vecino] == -1 && red[iArista].flow < red[iArista].capacity) 
			{
				nivel[vecino] = nivel[actual] + 1;
				nuevosVecinos.push_back(vecino);
			}
		}
		if (vecinos.empty())
		{
			swap(vecinos,nuevosVecinos);
			nuevosVecinos = {};
		}
	}
	return (nivel[t] != -1);
}

tint dfs (tint actual, tint flujo)
{
	if (flujo <= 0)
		return 0;
	else if (actual == t)
		return flujo;
	else
	{
		while (ultimoVecino[actual] < tint(ladj[actual].size()))
		{
			tint id = ladj[actual][ultimoVecino[actual]];
			if (nivel[red[id].end] == nivel[actual] + 1)
			{
				tint pushed = dfs(red[id].end,min(flujo,red[id].capacity-red[id].flow));
				if (pushed > 0)
				{
					red[id].flow += pushed;
					red[id^1].flow -= pushed;
					return pushed;
				}
			}
			ultimoVecino[actual]++;
		}
		return 0;	
	}
}

tint dinic ()
{
	tint flujo = 0;
	while (bfs())
	{
		
		forn(i,n+1)
			ultimoVecino[i] = 0;
		tint pushed = dfs(s,INFINITO);
		
		while (pushed > 0)
		{
			flujo += pushed;
			pushed = dfs(s,INFINITO);
		}
	}
	return flujo;
}

\end{code}

\subsection{Flujo de Costo M�nimo}

\begin{code}

\end{code}

\pagebreak

\section{Arboles}

\subsection{Union-Find (Guty)}
\begin{code}
const tint maxN = 131072;
vector<tint> caminito;
tint representante[maxN];
tint tamanho[maxN];

void inicializar (tint n)
{
	forn(i,n)
	{
		representante[i] = i;
		tamanho[i] = 1;
	}
}

tint find (tint x)
{
	caminito = {};
	while (x != representante[x])
	{
		caminito.push_back(x);
		x = representante[x];	
	}
	for (auto z : caminito)
		representante[z] = x;
	return x;
}

bool same (tint a, tint b)
{
	return (find(a) == find(b));
}

void unite (tint a, tint b)
{
	a = find(a);
	b = find(b);
	if (tamanho[a] < tamanho[b])
		swap(a,b);
	tamanho[a] += tamanho[b];
	representante[b] = a;	
}
\end{code}

\subsection{LCA - Segment Tree (Jonaz)}

\begin{code}

\end{code}

\subsection{Binary Lifting (saltitos potencia de 2)}

\begin{code}
const tint maxN = 32768; // cantidad de nodos
const tint maxK = 16;	 // lg(cantidadDeNodos)
const tint NEUTRO = 1e8; // neutro de la operacion (ejemplo: minimo)

tint d[maxN]; // profundidad
pair<tint,tint> p[maxN][maxK]; // {ancestro a distancia 2^k, 
																	Lo que queremos entre los 2^k ancestros}

void dfs(tint actual, vector<vector<pair<tint,tint> > > &ladj, tint padre)
{
	d[actual] = d[padre]+1;
	for (auto x : ladj[actual])
		if (x.first != padre)
		{
			p[x.first][0] = {actual,x.second};
			dfs(x.first,ladj,actual);
		}
}

tint subir(tint a, tint c, tint &ans, bool tomaMinimo)
{
	tint k = 0;
	while (c > 0)
	{
		if (c % 2)
		{
			if (tomaMinimo)
				ans = min(ans,p[a][k].second);
			a = p[a][k].first;
		}
		k++;
		c /= 2;
	}
	return a;
}


tint answer (tint a, tint b)
{
	// IGUALAMOS LAS ALTURAS
	if (d[a] < d[b])
		swap(a,b);
	tint w = d[a] - d[b], ans = NEUTRO;
	a = subir(a,w,ans,true);
	
	// HACEMOS LA BINARY PARA BUSCAR EL LCA
	tint cInf = 0, cSup = maxN; 
	while (cSup - cInf > 1)
	{
		tint ra = a, rb = b;
		tint c = (cSup+cInf)/2;
		ra = subir(ra,c,ans,false);
		rb = subir(rb,c,ans,false);
		if (ra == rb)
			cSup = c;
		else
			cInf = c;
	}
	// SUBIMOS LO QUE HAGA FALTA PARA LLEGAR AL LCA
	cSup *= (a != b);
	a = subir(a,cSup,ans,true);
	b = subir(b,cSup,ans,true);
	return ans;
}

// INICIALIZACION
int main()
{
	forn(i,maxN)
	forn(k,maxK)
		p[i][k] = {-1,NEUTRO};
	// HACEMOS EL PRIMER PASO EN FUNCION DEL GRAFO
	vector<vector<pair<tint,tint> > > ladj (maxN); // listaDeAdyacencia del arbol
	d[0] = -1;
	dfs(0,ladj,0);
	// LLENADO DE LA TABLA
	forsn(k,1,maxK)	
	forn(i,maxN)
	{
		tint ancestro = p[i][k-1].first;
		if (ancestro >= 0)
			p[i][k] = {p[ancestro][k-1].first, 
					       min(p[i][k-1].second,p[ancestro][k-1].second) };
	}
}
\end{code}

\pagebreak

\section{Strings}

\subsection{ToString/ToNumber}
\begin{code}
#include <iostream>
#include <string>
#include <sstream>

tint toNumber (string s)
{
	tint Number;
	if ( ! (istringstream(s) >> Number) )
		Number = 0; // el string vacio lo manda al cero
	return Number;
}

string toString (tint number)
{    
    ostringstream ostr;
    ostr << number;
    return  ostr.str();
}
\end{code}

\subsection{Tablita de Bordes (Jonaz)}

\begin{code}

\end{code}

\subsection{Knuth-Morris-Pratt (KMP) (Jonaz)}

\begin{code}

\end{code}

\subsection{Subsecuencia Comun mas larga (Guty o Jonaz)}

\begin{code}

\end{code}

\subsection{Edit-Distance}

\begin{code}
 // Minima distancia entre strings si lo que se puede es: INSERTAR, REMOVER, MODIFICAR, SWAPS ADYACENTES
const tint maxN = 1024; // maximo largo de los strings
const tint INFINITO = 1e15;
string s1,s2;
tint dist[maxN][maxN];

tint f(tint i, tint j)
{
	// Si un string es vacio, hay que borrar todo el otro
	if (i == -1 or j == -1) 
		return max(i,j)+1;
	if (dist[i][j] == INFINITO)
	{
		tint mini = INFINITO;		
     	// Lo mejor de borrar el i-esimo de s1 o insertar al final de s1 a s2[j]
		mini = min(mini,min(f(i-1,j)+1,f(i,j-1)+1));
		if (s1[i] == s2[j]) // Si coinciden, dejo como esta y resuelvo lo  anterior
			mini = min(mini,f(i-1,j-1)); 
		else // Modificar s1[i] a s2[j] y resolver lo anterior
			mini = min(mini,f(i-1,j-1)+1); 
			
		// Borramos los intermedios y swapeamos los ultimos 2 si funciona, lo hago y resuelvo lo anterior	
		forn(k,i) 
		{
			if (i >= 1 && j >= 1 && s1[i] == s2[j-1] && s1[i-k-1] == s2[j])
				mini = min(mini,f(i-k-2,j-2)+k+1);
		}
		dist[i][j] = mini;
	}
	return dist[i][j];
}
// USO: 
int main()
{
	tint n = s1.size(), m = s2.size();
	forn(i,n)
	forn(j,m)
		dist[i][j] = INFINITO;
	cout << f(n-1,m-1) << "\n";	
	return 0;
}
\end{code}

\subsection{Substring Palindromo (esPalindromo(s[i..j]))}

\begin{code}
// Asumo i < j
bool esPalindromo (tint i, tint j, vector<vector<tint> > &r, tint n)
{
	if (i+j >= n)
		return (r[i+j][n-i] - r[i+j][n-j-1]) == j-i+1;
	else
		return (r[i+j][j+1] - r[i+j][i]) == j-i+1;
}
// USO:
int main()
{
		tint n = s.size(); // s nuestro string
		vector<vector<tint> > v (n, vector<tint> (n,0));
		forn(i,n)
		forn(j,n)
			v[i][j] = (s[i] == s[j]);
		vector<vector<tint> > r (2*n-1,vector<tint> (n+1,0));
		
		forn(i,2*n-1)
		{
			tint sum = 0, x = min(i,n-1), y = 0;
			if (i >= n)
				y = i-n+1;
			forn(j,n)
			{
				if (x >=0 && y < n)
					sum += v[x--][y++];
				r[i][j+1] = sum;
			}
		}
		// Ahora podemos preguntar si es palindromo s[i..j]
}
\end{code}

\subsection{Suffix Array}

\begin{code}

\end{code}

\subsection{Longest Common Prefix}

\begin{code}

\end{code}

\pagebreak

\section{Matematica}

\subsection{PotLog}

\begin{code}
const tint nmod = 1000000007; // o el primo que deseamos
tint potLogMod (tint x, tint y) // Calcula: (x^y) mod nmod
{
	tint ans = 1;
	while (y > 0)
	{
		if (y % 2)
			ans = (x * ans) % nmod;
		x = (x * x) % nmod;
		y /= 2;
	}
	return ans;
}
tint invMod(tint a) // nmod PRIMO. Devuelve b tal que: (a*b) = 1 (mod nmod) 
{
	return potLogMod(a,nmod-2);
}
\end{code}

\subsection{Criba}

\begin{code}
const tint maxN = 1000500;
tint p[maxN + 1] = {1, 1};
tint phi[maxN]; 

map<tint,tint> factorizar (tint n)
{
	map<tint,tint> f;
	while (n > 1) 
	{ 
		f[p[n]]++;
		n /= p[n]; 
	}
	return f;
}
// USO:
int main()
{
	// CRIBA COMUN : (p[n] = mayor primo que divide a n (n >= 2) )
	for (tint i = 1; i <= maxN; ++i)
		if (p[i] == 1)
			for (tint j = i; j <= maxN; j += i)
				//if (p[j] == 1 or i == 1) // Con esta linea da el menor primo
				p[j] = i;
	// CALCULA PHI(N): #Coprimos con N
	for (tint i = 0; i < maxN; i++) 
		phi[i] = i;	
	for (tint i = 1; i < maxN; i++)
	for (tint j = 2 * i; j < maxN; j += i)
		phi[j] -= phi[i];
	return 0;
}
\end{code}

\subsection{Euclides Extendido (Guty)}

\begin{code}

\end{code}

\subsection{Teorema Chino del Resto (Guty)}

\begin{code}

\end{code}

\subsection{Eliminacion Gaussiana}

\begin{code}

\end{code}

\subsection{Rabin-Miller}

\begin{code}

// USA: "PotLog", pero pasandole el modulo como parametro
#include <random>
const tint semilla = 38532164;
mt19937 gen(semilla);

tint mult(tint a, tint b, tint m)
{
	int largestBit = 0;
	while( (b >> largestBit) != 0)
		largestBit++;
	tint ans = 0;
	for(tint currentBit = largestBit - 1; currentBit >= 0; currentBit--)
	{
		ans = (ans + ans);
		if (ans >= m)
			ans -= m;

		if ( (b >> currentBit) & 1)
		{ 
			ans += a;
			if (ans >= m)
				ans -= m; 
		}
	}
	return ans;
}

bool esPrimoRM (tint n)
{
	if (n <= 1)
		return false;
	else if (n <= 3)
		return true;
	else if (n % 2 == 0)
		return false;
	else
	{
		uniform_int_distribution<tint> dis(2, n-2);
		tint kOrig = 0, m = n-1;
		while (m % 2 == 0)
		{
			kOrig++;
			m /= 2;
		}
		bool esPrimo = true;
		vector<tint> testigos = {2,3,5,7,11,13,17,19,23,29,31,37};
		for (auto a : testigos)
		{
			if (a < n)
			{
				tint b = potLogMod(a,m,n), k = kOrig;
				if (b == 1 or b == n-1)
					continue;
				else
				{
					forn(j,k)
					{
						b = mult(b,b,n);
						if (b == n-1)
							break;
						else if (b == 1)
						{
							esPrimo = false;
							break;
						}
					}
					if (b != n-1)
					{
						esPrimo = false;
						break;
					}
				}
			}
		}
		return esPrimo;
	}
}
\end{code}

\subsection{Pollard-Rho}

\begin{code}
// USA: Rabin-Miller
tint gcd (tint a, tint b)
{
	if (a == 0)
		return b;
	return gcd (b % a, a);
}
void factorizar (tint n, map<tint,tint> &f)
{
	while (n > 1)
	{
		if (esPrimoRM(n))
		{
			f[n]++;
			n /= n;
		}
		else
		{
			uniform_int_distribution<tint> dis(1, n-1);
			tint a = dis(gen), b = dis(gen), x = 2, y = 2, d;
			do
			{
				x = (mult(x,x,n) + mult(a,x,n) + b) % n;
				y = (mult(y,y,n) + mult(a,y,n) + b) % n;
				y = (mult(y,y,n) + mult(a,y,n) + b) % n;
				d = gcd(abs(x-y),n);
			}
			while (d == 1);
			if (d != n)
			{
				factorizar(d,f);
				n /= d;
			}
				
		}
	}
}

\end{code}

\subsection{FFT}

\begin{code}
// USA : "PotLog" e "InvMod" con nmod = mod 
const tint mod = (1 << 21)*11 + 1 ; // es re primo
const tint root = 38;
const tint root_1 = 21247462;
const tint root_pw = 1 << 21 ; // largo del arreglo
/*
 * const tint mod = 7340033;
 * const tint root = 5 ;
 * const tint root_1 = 4404020 ;
 * const tint root_pw = 1 << 20 ; 
 */
 
tint modulo (tint n)
{
	return ((n % mod) + mod) % mod;
}
void fft (vector <tint> &a, bool invert )
{
	tint n = a. size();
	for (tint i = 1 , j = 0 ; i < n ; ++ i )
	{
		tint bit = n >> 1 ;
		while(j >= bit)
		{
			j -= bit ;
			bit >>= 1;
		}
		j += bit ;
		if ( i < j )
			swap (a[i],a[j]);
	}
	for (tint len = 2 ; len <= n ; len <<= 1) 
	{
		tint wlen = root;
		if (invert)
			wlen = root_1;
		for (tint i = len ; i < root_pw ; i <<= 1)
			wlen = modulo(wlen * wlen);
		for (tint i = 0 ; i < n ; i += len ) 
		{
			tint w = 1 ;
			forn(j,len/2)
			{
				tint u = a[i+j], v = modulo(a[i+j+len/2] * w) ;
				a[i+j] = modulo(u+v);
				a[i+j + len/2] = modulo(u - v);
				w = modulo(w * wlen) ;
			}
		}
	}
	
	if (invert) 
	{
		tint nrev = invMod(n);
		forn(i,n)
			a[i] = modulo(a[i] * nrev) ;
	}
}

void multiply (const vector<tint> &a, const vector<tint> &b, vector<tint> &res) 
{
	vector<tint> fa(a.begin(), a.end() ), fb(b.begin(), b.end() );
	tint n = 1 ;
	while (n < max(tint(a.size()), tint(b.size())))
		n <<= 1;
	n <<= 1 ;
	fa.resize(n), fb.resize(n);
	fft (fa, false) , fft(fb, false);
	forn(i,n)
		fa[i] *= fb[i];
	fft(fa, true);
	res = fa;
} 
// USO:
int main()
{
	vector<tint> a = {1,0,0,1};
	vector<tint> b = {1,0,0,1};
	vector<tint> res;
	multiply(a,b,res);
	for (auto x : res)
		cout << x << " " ; // 1 0 0 2 0 0 1 0 
	cout << endl;
	return 0;
}

\end{code}

\subsection{Regla de Simpson (Integracion Numerica) (Guty)}

\begin{code}

\end{code}

\pagebreak

\section{Geometria}

\subsection{Tipo PUNTO y Operaciones}

\begin{code}
const ldouble epsilon = 1e-10;
const ldouble pi = acos(-1);

struct Punto
{
	ldouble x,y;
	Punto (ldouble xx, ldouble yy)
	{
		x = xx;
		y = yy;
	}
	Punto()
	{
		x = 0.0;
		y = 0.0;
	}
};
Punto operator + (Punto p1, Punto p2)
{
	return Punto(p1.x+p2.x,p1.y+p2.y);
}
Punto operator - (Punto p1, Punto p2)
{
	return Punto(p1.x-p2.x,p1.y-p2.y);
}
Punto operator * (ldouble lambda, Punto p)
{
	return Punto(lambda*p.x, lambda*p.y);
}
ldouble operator * (Punto p1, Punto p2)
{
	return p1.x*p2.x+p1.y*p2.y;
}
ldouble operator ^ (Punto p1, Punto p2)
{
	return p1.x*p2.y - p1.y*p2.x;
}
Punto operator ~ (Punto p)
{
	return Punto(-p.y,p.x);
}
ldouble norma (Punto p)
{
	return sqrt(p.x*p.x+p.y*p.y);
}
bool operator < (Punto p1, Punto p2)
{
	return make_pair(p1.x,p1.y) < make_pair(p2.x,p2.y);
}
bool operator == (Punto p1, Punto p2)
{
	return ((abs(p1.x-p2.x) < epsilon) && (abs(p1.y-p2.y) < epsilon));
}
\end{code}


\subsection{Area de Poligono}

\begin{code}
ldouble areaTriangulo (Punto p1, Punto p2, Punto p3)
{
	return abs((p1-p3)^(p1-p2))/2.0;
}

ldouble areaPoligono(vector<Punto> &polygon)
{
	ldouble area = 0.0;
	tint n = polygon.size();
	forn(i,n)
		area += polygon[i]^polygon[(i+1)%n];
	return abs(area)/2.0;
}
\end{code}

\subsection{Punto en Poligono}

\begin{code}
bool adentroPoligono(vector<Punto> &polygon, Punto p) // polygon EN EL SENTIDO DE LAS AGUJAS
{
	bool adentro = true;
	tint n = polygon.size();
	forn(i,n)
		adentro &= (((p-polygon[i])^(p-polygon[(i+1)%n])) < 0);
	return adentro;
}
\end{code}

\subsection{Interseccion de Segmentos}

\begin{code}
struct Segmento
{
	Punto start,end,dir;
	Segmento (Punto ss, Punto ee)
	{
		start = ss;
		end = ee;
		dir = ee-ss;
	}
};
// res.second == 0 -> NO HAY INTERSECCION
// res.second == 1 -> INTERSECAN EN UN PUNTO (que esta en res.first)
// res.second == 2 -> SON COLINEALES E INTERSECAN EN TODO UN SEGMENTO (Da un extremo, si queremos el otro, correr otra vez con "otroExtremo" = true)
pair<Punto,tint> interSeg (Segmento s1, Segmento s2, bool otroExtremo ) 
{
	if ((abs(s1.dir ^ s2.dir)) < epsilon) // son colineales
	{
		vector<pair<Punto,tint> > aux = {{s1.start - epsilon*s1.dir,1},
																	 	 {s1.end   + epsilon*s1.dir,1},
															 			 {s2.start - epsilon*s2.dir,2},
															 			 {s2.end   + epsilon*s2.dir,2}};
		sort(aux.begin(),aux.end());
		if (aux[0].second != aux[1].second)
			return make_pair(aux[1+otroExtremo].first,2);
		else
			return make_pair(Punto(),0);
	}
	else
	{
		ldouble alfa = ((s2.start-s1.start)^s2.dir) / (s1.dir^s2.dir);
		if (0 <= alfa && alfa <= 1)
			return make_pair(s1.start+alfa*s1.dir,1);
		else
			return make_pair(Punto(),0);
	}
}

\end{code}

\subsection{Angulo Entre Puntos y Distancia entre Segmentos}
\begin{code}
ldouble angEntre (Punto p1, Punto p2, Punto p3) // P1^P2P3
{
	ldouble a = norma(p2-p3);
	ldouble b = norma(p1-p3);
	ldouble c = norma(p2-p1);
	return acos((a*a+c*c-b*b)/(2*a*c));
}

ldouble dPuntoSeg (Punto p, Segmento s)
{
	if (angEntre(p,s.start,s.end) > pi/2 or angEntre(p,s.end,s.start) > pi/2)
		return min(norma(p-s.start),norma(p-(s.end)));
	else
		return abs( ((s.start-p)^(s.end-p)) / (norma(s.dir)) );
}

ldouble dEntreSeg(Segmento s1, Segmento s2)
{
	ldouble a = min(dPuntoSeg(s1.start,s2),dPuntoSeg(s1.end,s2));
	ldouble b = min(dPuntoSeg(s2.start,s1),dPuntoSeg(s2.end,s1));
	return (interSeg(s1,s2,false).second == 0) * min(a,b);
}
\end{code}

\subsection{Convex-Hull (2D) (Jonaz)}

\begin{code}

\end{code}

\subsection{Sweep Line Facil (Interseccion de Segmentos/Closest Pair)}

\begin{code}

\end{code}

\subsection{Sweep Line Dificil (Union de Rect�ngulos)}

\begin{code}

\end{code}

\subsection{Radial Sweep}

\begin{code}

\end{code}

\subsection{Minimum Bounding Circle}

\begin{code}

\end{code}

\pagebreak

\section{Varios}

\subsection{Longest Incresing Subsequence (LIS)}

\begin{code}
tint LIS(vector<tint> &v) {
  if (v.empty()) return 0;
  
  tint l = 0;             // ultimo lugar de tails hasta ahora
  vi tails(v.size(), 0);  // candidatos de final de sub secuencias
  tails[l] = v[0];
  
  forsn(i,1,v.size()) {
    // con upper_bound es no-decreciente
    tint me = lower_bound(tails.begin(),tails.begin()+l+1, v[i])-tails.begin();
    tails[me] = v[i];
    if (me > l) l = me;
  }
  return l + 1;
}
\end{code}

\subsection{Maximum Subarray Sum}

\begin{code}

tint maximumSum (vector<tint> &a) // a no vacio
{
	tint maxTotal = a[0], maxAca = a[0], n = a.size();
	forsn(i,1,n)
	{
		maxAca = max(a[i],maxAca + a[i]);
		maxTotal = max(maxTotal,maxAca);
	}
	return maxTotal;
}
\end{code}

\subsection{Rotar 90� una matriz (sentido horario)}

\begin{code}
void rotar (vector<string> &origi) 
{
	tint n = origi.size();
	string aux (n,'x');
	vector<string> rotado (n,aux);
	forn(i,n)
	forn(j,n)
		rotado[j][n-i-1] = origi[i][j];
	origi = rotado;
}
\end{code}


\subsection{Random + Imprimir Doubles}

\begin{code}
#include <iostream>
#include <random>
#include <iomanip>

using namespace std;

random_device rd;
mt19937 gen(rd());
uniform_int_distribution<int> dis1(1, 10000);
uniform_real_distribution<long double> dis2(1, 10000);

int main()
{
	cout <<  dis1(gen) << "\n";
	cout <<  fixed << showpoint << setprecision(16) << dis2(gen) << "\n";
	return 0;
}
\end{code}

\subsection{Slding Window RMQ}

\begin{code}

void agrandarVentana (tint &r, deque<pair<tint,tint> > &rmq, vector<tint> &v)
{
	while (!rmq.empty() && rmq.back().first >= v[r])
		rmq.pop_back();
	rmq.push_back({v[r],r});
	r++;
	
}

void achicarVentana (tint &l, deque<pair<tint,tint> > &rmq)
{
	if (l == rmq.front().second)
		rmq.pop_front();
	l++;
}

pair<tint,tint> minimoVentana (deque<pair<tint,tint> > &rmq)
{
	return rmq.front();
}
// USO: En todo momento tenemos el minimo entre [l,r)
int main()
{
	deque<pair<tint,tint> > rmq; // {numero,indice}
	tint l = 0, r = 0;//  l   . r
	vector<tint> v = {1,2,3,4,3,2,2,3,4};
	agrandarVentana(r,rmq,v);
	agrandarVentana(r,rmq,v);
	agrandarVentana(r,rmq,v);
	agrandarVentana(r,rmq,v);
	agrandarVentana(r,rmq,v);
	achicarVentana(l,rmq);
	achicarVentana(l,rmq);
	cout << minimoVentana(rmq).first << endl;	// {3,4}
	return 0;
}
\end{code}

\subsection{Ternary Search}

\begin{code}
// Ternary en ENTEROS
tint miniTernarySearch (tint a, tint b) // En [a,b] esta el minimo
{
	tint l = a, r = b;
	while (abs(r - l) > 5)
	{
		tint al = (2*l + r)/3;
		tint br = (l + 2*r)/3;
		if (f(al) > f(br)) // cambiar a "<" para maximo
			l = al;
		else
			r = br;			
	}
	tint ans = 1e16;
	forsn(k,l,r+1)
		ans = min(ans,f(k));  // cambiar por "max" para maximo
	return ans;
}
//Ternary en FLOATING POINT
ldouble miniTernarySearch (ldouble tL, ldouble tR) // En [tL, tR] esta el minimo
{
	while (abs(tR - tL) > epsilon)
	{
		ldouble tLThird = (2.0*tL + tR)/3.0;
		ldouble tRThird = (tL + 2.0*tR)/3.0;
		if (f(tLeftThird) > f(tRightThird)) // cambiar a "<" para maximo
			tLeft = tLeftThird;
		else
			tRight = tRightThird;			
	}
	return f((tLeft+tRight)/2.0);
}

\end{code}







\end{document}
